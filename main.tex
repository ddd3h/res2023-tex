\RequirePackage{plautopatch}
\documentclass[uplatex,a4paper,11pt,dvipdfmx]{jsreport}

\usepackage{bm}
\usepackage{physics}
\usepackage[dvipdfmx]{graphicx}
\usepackage{here}
\usepackage[utf8]{inputenc}
\usepackage{tcolorbox}
\tcbuselibrary{breakable,theorems}
\usepackage{tikz}
\usetikzlibrary{intersections, calc, arrows.meta}
\usepackage{amsmath}
\usepackage{titleps}
\usepackage{enumerate}
\usepackage{amsfonts}
\usepackage{amssymb}
\usepackage{wrapfig}
\usepackage{ascmac}
\usepackage{siunitx}
\usepackage{cancel}
% \usepackage{udline}
\usepackage[hang,small,bf]{caption}
\usepackage[subrefformat=parens]{subcaption}
\usepackage{import}
\usepackage{color}
\usepackage{okumacro}
\usepackage{framed}
\usepackage[dvipdfmx]{hyperref}

\usepackage{algpseudocode}
\usepackage{algorithm}


\setlength{\textwidth}{\fullwidth}  %本文の幅(textwidth)を全体の幅(=ヘッダ部の幅)にそろえる
\setlength{\evensidemargin}{\oddsidemargin} %偶数ページの余白と奇数ページの余白をそろえる

\title{\textbf{宇宙論的シミュレーションデータベースIllustris-TNGを用いた銀河周辺物質の速度と元素分布構造の解明}}
\author{埼玉大学理学部物理学科 \\ 宇宙物理実験研究室 \\ \\ 20RP021 西濱大将}
\date{2024/02/15}

\begin{document}
    \maketitle
    \tableofcontents
    \begin{abstract}
    あ
\end{abstract}
    \chapter{序章}
\section{ああ}

    人間は、長い進化の過程を経て、この地球上で最も進化した生物となりました。私たちは、知能や言語能力、社会性など、多くの特徴を持ち、自然界の中でも特別な存在となっています。

    人間は、自分たちが生きる環境を変えることができる唯一の生物です。私たちは、火を使ったり、農業を始めたり、建築物を作ったりすることで、自分たちの生活を改善し、より豊かな生活を送ることができるようになりました。

    しかし、人間が自分たちの生活を改善するために行うことには、悪影響もあります。工場の排気や自動車の排気ガスなど、人間が生み出す大気汚染物質は、大気中のオゾン層を破壊し、地球温暖化を引き起こす原因となっています。

    また、人間が大量に生産するために必要な資源を得るために、森林破壊や海洋汚染なども引き起こしています。これらの問題は、地球全体の生態系に深刻な影響を与える可能性があります。

    人間は、これらの問題に対処するために多くの努力をしています。再生可能エネルギーの開発や、環境に配慮した建築物の建設、リサイクルなど、多くの取り組みが進められています。

    しかし、これらの問題を解決するためには、個人の行動も重要です。例えば、自動車をできるだけ使わずに、公共交通機関を利用することや、家庭でのエネルギー使用量の削減などが、地球環境を守るためには必要なことです。

    近年、AI技術の進歩により、人間の生活や社会にも大きな変化が起きています。自動運転車の開発や、ロボット技術の進歩、自然言語処理の発展など、多くの分野でAI技術が活用されています。

    しかし、AI技術の進歩には、倫理的な問題も付きまといます。例えば、AIが人間の判断を代替することで、倫理的な問題を引き起こす可能性があります。また、AIが偏った情報を学習することで、社会的な偏見を強化することもあります。

    これらの問題に対処するためには、AI技術の開発にあたって、倫理的な観点が重視される必要があります。AI技術を開発する企業や研究者は、社会的責任を持ち、偏見の排除や倫理的な問題に対処することが求められます。

    人間は、今後も進化し続け、新たな課題に直面することになるでしょう。しかし、私たちが持つ知能や創造性を活用し、倫理的な観点から考えた行動を取ることで、より持続可能な社会を築くことができると信じられています。私たちは、地球上で生きるために必要な資源を守り、環境に配慮した持続可能な社会を作ることが大切です。

    また、人間は、単なる知的生物ではありません。私たちは、感情や思いやりを持ち、他者と共感し、共に生きることができます。この人間の特性を活かし、社会全体で協力し、互いに支え合うことで、より良い社会を作ることができるでしょう。

    最後に、人間は、この地球上で唯一の進化した生物です。私たちは、自分たちが生きる環境を守り、持続可能な社会を作ることが求められています。私たちは、自分たちが生きる世界をより良くしていくために、今後も知恵を絞り、努力し続ける必要があるでしょう。
    \chapter{手法}

\section{ビリアル半径 $R_\text{vir}$}

赤方偏移$z$においてビリアル平衡に達したダークマターハローの平均密度は:
\begin{align}
	\rho_\text{vir}(z) &= \rho_\text{cr} \Delta_\text{vir} \\
	&\simeq \num{1.8e-27} \left( \frac{\Delta_\text{vir}}{200}\right) \left(\frac{h}{0.7}\right)^2 E^2(z) \ \si{g. cm^{-3}}
\end{align}
と表せる。

ここでビリアル半径内の全質量(ビリアル質量)を$M_\text{vir}$を用いると
\begin{align}
	R_\text{vir} &= \left( \frac{3 M_\text{vir}}{4 \pi \rho_\text{vir}(z)} \right)^{1/3} \\
	&\simeq 2.1 \left( \frac{M_\text{vir}}{10^{15} M_\odot} \right)^{1/3} \left( \frac{\Delta_\text{vir}}{200} \right)^{-1/3} \left( \frac{h}{0.7} \right)^{-2/3} E^{-2/3}(z) \quad \si{Mpc}
\end{align}
となり、観測される銀河団のサイズと質量等の関係を近似的に再現する。

ここでは近傍宇宙を考えているので、$z=0$では$E(z) = 1$であり、$\Delta_\text{vir} = 200$のときのビリアル半径を$R_{200}$とすると
\begin{align}
	R_{200} \simeq 2.1 \left( \frac{M_\text{vir}}{10^{15} M_\odot} \right)^{1/3} \left( \frac{h}{0.7} \right)^{-2/3} \quad \si{Mpc}
\end{align}

\begin{algorithmic}
	\Function{Solve\_Virial\_Mass}{$\text{radius}, \text{mass}, \text{density\_DM}, \text{density\_total}$}
	\State $\text{valid\_indices} \gets \text{indices of } \text{radius} \text{ where } \text{total\_mass} \text{ is not } \infty$
	\State $\text{radius} \gets \text{radius}[ \text{valid\_indices}]$
	\State $\text{total\_mass} \gets \text{total\_mass}[ \text{valid\_indices}]$
	
	\State $\text{sorted\_radius}, \ \text{sorted\_total\_mass} \gets \text{sort } \text{radius}, \ \text{total\_mass} \text{ based on } \text{radius}$
	
	\State $\text{cum\_mass} \gets \text{cumulative sum of } \text{sorted\_total\_mass}$
	
	\State $h \gets 0.6774$
	
	\State $\text{virial\_radius} \gets 2.1 \times \left(\frac{\text{cum\_mass} \times 10^{10}}{10^{15}}\right)^{\frac{1}{3}} \times \left(\frac{h}{0.7}\right)^{-\frac{2}{3}}$
	
	\State $\text{min\_index} \gets \text{index of the element in } \text{radius}$ \\
	\qquad \qquad \qquad \qquad \qquad$\text{ with the smallest value of } (\text{radius} - \text{virial\_radius})^2$
	\State $ r_{200} \gets \text{radius}[\text{min\_index}]$
	
	\State \textbf{return} $r_{200}$
	\EndFunction
\end{algorithmic}
    \documentclass[main.tex]{subfiles}
\begin{document}
	\chapter{結果}
	
	\section{アウトフローの確認}
	subhalo342447について図\ref{fig:outflowsubhalo342447}に示すようにアウトフローを確認した.一方でsubhalo388544はアウトフローが確認できず,subhalo421555は下向きの片方のみ確認できた (図\ref{fig:outflowsubhalo}).
	
	
	\begin{figure}[htbp]
		\centering
		\includegraphics[width=0.6\linewidth]{pic/outflow_subhalo342447}
		\captionsetup{width=.8\linewidth}
		\caption{subhalo342447をedge-onで表示したもの.Massesを背景に表示して速度場を表示している.}
		\label{fig:outflowsubhalo342447}
	\end{figure}
	
	\begin{figure}[htbp]
		\centering
		\begin{minipage}[b]{0.45\linewidth}
			\centering
			\includegraphics[width=\linewidth]{pic/outflow_subhalo388544}
			\subcaption{subhalo388544}
			\label{fig:outflowsubhalo388544}
		\end{minipage}
		\begin{minipage}[b]{0.45\linewidth}
			\centering
			\includegraphics[width=\linewidth]{pic/outflow_subhalo421555}
			\subcaption{subhalo421555}
			\label{fig:outflowsubhalo421555}
		\end{minipage}
		\captionsetup{width=.9\linewidth}
		\caption{edge-onで表示.Massesを背景に表示して速度場を表示している.}
		\label{fig:outflowsubhalo}
	\end{figure}
	
	\section{動径方向の元素分布}
	
	subhao342447の動径方向の元素分布を図\ref{fig:abundanceprofile342447}に示す.R/R$_{200} < 0.1$においてsolar abundanceは,2倍程度であり,中心から外縁に向かうに連れて現象していった.
	
	\begin{figure}[htbp]
		\centering
		\includegraphics[width=0.6\linewidth]{pic/abundance_profile342447}
		\captionsetup{width=\linewidth}
		\caption{subhalo342447のFe, O, Mg, Siの太陽組成比を縦軸にし,横軸をsubhaloの中心(subhaloの移動質量中心で,subhalo内のすべての粒子/セルの質量加重相対座標の合計として計算される)からの距離をビリアル半径で規格化したものを片対数で表している.データは15個にグループまとめし,エラーバーは横軸がデータ幅(上限と下限),縦軸が標準偏差を表す.}
		\label{fig:abundanceprofile342447}
	\end{figure}
	
	一方でsubhalo388544とsubhalo421555においてR/R$_{200} < 0.1$では,太陽組成程度で中心から外縁部に向かって減少していることが分かった(図\ref{fig:2radicalprofile}).
	
	\begin{figure}[htbp]
		\centering
		\begin{minipage}[b]{0.45\linewidth}
			\centering
			\includegraphics[width=\linewidth]{pic/abundance_profile388544}
			\subcaption{subhalo388544}
			\label{fig:abundanceprofile388544}
		\end{minipage}
		\begin{minipage}[b]{0.45\linewidth}
			\centering
			\includegraphics[width=\linewidth]{pic/abundance_profile421555}
			\subcaption{subhalo421555}
			\label{fig:abundanceprofile421555}
		\end{minipage}
		\caption{太陽組成比-対-ビリアル半径で規格化した半径.図\ref{fig:abundanceprofile342447}と同様}
		\label{fig:2radicalprofile}
	\end{figure}
	
	subhalo342447において,鉄の太陽組成比に対する酸素,ネオン,マグネシウムとケイ素の太陽組成比を計算したものを図\ref{fig:fe10}に示す.$\text{[O/Fe]} \sim 0.33$程度,$\text{[Ne/Fe]} \sim 0.53$程度,$\text{[Mg/Fe]} \sim 0.23$程度,$\text{[Si/Fe]} \sim 0.2$程度であった.図\ref{fig:fe10}では$0.1<\text{R/R}_{200} < 0.5$で他の部分と組成が異なる「へこみ」が観測された.
	
	\begin{figure}[htbp]
		\centering
		\includegraphics[width=0.8\linewidth]{pic/Fe10}
		\caption{subhalo342447の[X/Fe] (ただし X $=$O, Ne, Mg, Si)を縦軸にし,横軸をsubhaloの中心(subhaloの移動質量中心で,subhalo内のすべての粒子/セルの質量加重相対座標の合計として計算される)からの距離をビリアル半径で規格化したものを片対数で表している.データは15個にグループまとめし,エラーバーは横軸がデータ幅(上限と下限),縦軸が標準偏差を表す.}
		\label{fig:fe10}
	\end{figure}
	
	\section{温度分布}
	
	subhalo342447の温度分布を図\ref{fig:atemp}に示す.図\ref{fig:atemp}の右側で,R/R$_{200} < 0.1$においては\SI{e6}{K}以上の高温ガスが確認できた.左側は\num{1e+5}から\SI{4e+5}{K}の範囲の比較的低い温度が観測された.温度分布は左右で非対称であることが確認できる.
	
	また明らかに銀河の腕部分は\SI{e5}{K}以下と非常に低温なガスであることが確認できる.
	
	\begin{figure}[htbp]
		\centering
		\includegraphics[width=0.8\linewidth]{pic/a_temp}
		\captionsetup{width=0.9\linewidth}
		\caption{face-on表示にしたsubhalo342447を各メッシュ内の平均温度を表している.円はビリアル半径を表す.計算にはガスのみを対象とした.}
		\label{fig:atemp}
	\end{figure}

\end{document}

    \documentclass[main.tex]{subfiles}
\begin{document}
	\chapter{議論}
	\section{subhalo342447における$\mathrm{[X/Fe]} > 0$について}

図\ref{fig:fe10}に示すように,X $=$ Ne, O, Mg, Siとして$\mathrm{[X/Fe]} > 0$がいえる.このようなXをアルファ($\alpha$)元素という.中性子数と陽子数が偶数で等しく,$\alpha$粒子(\ce{^{4}He})の集まりと見なせることから,この名前で呼ばれている.Feよりもアルファ($\alpha$)元素が多いということは,アルファ($\alpha$)元素の生成元される重力崩壊型超新星爆発(II型超新星爆発)に由来するものと考えられる.

II型超新星の中心部では核融合反応が進行し,アルファ($\alpha$)元素を生成する.核融合のエネルギーと重力が平衡状態であったのが,鉄まで生成されると平衡状態が崩れ,収縮を始める.中心核は中性子の縮退圧と重力が枯渇すると急停止し,上層は中心核によって反跳し衝撃波が発生する.ゆえに大量のアルファ($\alpha$)元素を宇宙空間にばらまくことになる.

\section{アウトフローと太陽組成比との関係}

アウトフローが観測されたsubhalo342447は図\ref{fig:abundanceprofile342447}に示したように,R/R$_{200} < 0.1$において太陽組成の2倍近くFe, O, Mg, Siなどの元素が観測された.一方でアウトフローが観測されなかったsubhalo388544やsubhalo421555はR/R$_{200} < 0.1$において太陽組成程度であることが観測された(図\ref{fig:2radicalprofile}).

このようなことからアウトフローと太陽組成には何らかの因果関係がある可能性がある.

\section{subhalo342447の「へこみ」と温度分布の高温部}

図\ref{fig:fe10}において$0.1<\text{R/R}_{200} < 0.5$で他の部分と組成が異なる「へこみ」が観測されたことに加え,図\ref{fig:atemp}の右側で,R/R$_{200} < 0.1$においては\SI{e6}{K}以上の高温ガスが確認できたことから,subhalo342447が現在の状態になるまでに,他のsubhaloと衝突をし,そのsubhaloの組成の一部を取り込んだ可能性が考えられる.

\begin{figure}[htbp]
	\centering
	\includegraphics[width=0.6\linewidth]{pic/virial3}
	\captionsetup{width=0.9\linewidth}
	\caption{subhalo342447をedge-onにした状態で中心にとり,ビリアル半径の3倍を表示.濃淡は対数で質量を表す.}
	\label{fig:virial3}
\end{figure}

また図\ref{fig:virial3}に示すようにsubhalo342447の周囲に別のsubhaloが観測され,ガスの散乱具合も衝突後のような状態となっている.このようなことから衝突した可能性は非常に高いと考えられる.
\end{document}
    \documentclass[main.tex]{subfiles}
\begin{document}
	\chapter{まとめと展望}
	今回主に解析をしたsubhalo342447は動径方向の元素分布や温度分布から重力崩壊型の超新星爆発に由来し,形成過程において他のsubhaloと衝突した可能性があることが示された.またアウトフローと太陽組成比に何らかの因果関係がある可能性があることも分かった.
	
	今後はsubhalo342447を実際にXRISM衛星\footnote{
	XRISM(クリズム)は、星や銀河の間を吹き渡る高温プラズマを観測するX線天文衛星です。2016年に姿勢制御系の不具合のため短期間で運用終了したX線天文衛星「ひとみ」の後継機として開発され,JAXA宇宙科学研究所,NASA,ESAといった国内外の研究機関が共同開発した衛星である.2023年9月7日に打ち上げられ,約4ヶ月にわたる立ち上げ運用を終えて,本格観測を開始した.
	
	XRISMは,搭載する最新のX線分光装置やX線撮像装置で宇宙を観測し,高温プラズマの速度や化学組成を調べる.X線の精密分光撮像によって,星や銀河の中でつくりだされる物質やエネルギーの流転を調べ,天体の進化を解明する.
	
	XRISMは,これまでにない高い精度でX線を検出することができ,データから天体の温度や速度などの状態を分析することで,ブラックホールや暗黒物質=ダークマターなど宇宙の謎の解明につながる可能性があると期待されている.}で観測した場合,どのようなデータが得られるのかを調査したい.またsubhalo342447のような銀河を実際にアウトフローや物質構造を推定するためにはどれぐらいの分解能を有する検出器が必要で,なおかつ必要な観測時間はどれほどであるかを計算することで今後の検出器の開発目標値としての指針を与えることができるであろう.

	さらに,subhalo342447以外のsubhaloについても同様の解析を行い,複数のデータを比較することで一般的な傾向や特異性を把握していきたい。

	このような詳細な解析と観測により,銀河の形成過程や物質の分布,そしてアウトフローと太陽組成比の関係性についての理解が深まり,宇宙の起源や進化についての新たな知見を提供し,我々の宇宙観を一段と進化させる可能性があるだろう.
	
\end{document}
    
\end{document}