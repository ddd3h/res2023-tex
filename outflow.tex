\section{速度場の計算}

アウトフローを確認するにあたり,速度場を描画する.その際の速度場の描画方法について説明する(図\ref{fig:calc_outflow}).約1億個以上の粒子(ガス)をそれぞれの速度を表示するのは無謀であるため,空間をある一定の数で分ける.経験的に20や30といった値で$x$軸,$y$軸のそれぞれを分割するのが良い.

その分けられたメッシュ内の平均速度$\bm{v}=(v_x,v_y,v_z)$を計算する.ただし表示は$xy$平面で表示を行っているが,粒子(ガス)は3次元情報であるため,$z$軸の方向に対しても平均をとっていることに注意が必要である.すなわち,アウトフローを確認するにあたりあまりにも大きくデータを表示すると,アウトフロー以外の部分についても平均を計算する際に加味され,正確にアウトフローを確認することができない.そのためsubhalo中心付近で表示して再確認をするなどの工夫が必要である.
\begin{figure}
	\centering
	
	\begin{minipage}[b]{0.33\linewidth}
		\centering
		\includegraphics[width=\linewidth]{./pic/outflow_ex4.png}
		\subcaption{}
		\label{}
	\end{minipage}
	\begin{minipage}[b]{0.45\linewidth}
		\centering
		\includegraphics[width=\linewidth]{./pic/outflow_ex5.png}
		\subcaption{}
		\label{}
	\end{minipage}
	
	\caption{}
	\label{fig:calc_outflow}
\end{figure}

本論文においては,subhalo中心付近のみ射影(プロジェクション)を行い,なおアウトフローが確認できなかったものについて「アウトフローが確認できなかった」と表現している.