\subsection{渦巻銀河}

渦巻銀河は記号Sの後にa, b, cをつけて分類される.可視光で見た渦巻銀河はバルジと呼ばれる中心の回転楕円体状の成分と広がった円盤成分からなる.円盤では回転運動が卓越しているが,バルジではランダムな運動が卓越している.円盤ではガスや塵が多く,星形成活動が活発である.このガスと塵,およびそれから生まれたばかりの若い星は円盤の赤道面の薄い層に強く集中しており,渦巻腕として顕著に見える.

また密度が大変低いので通常の画像では見えにくいが,円盤よりもさらに遠くまで広がり,ほぼ球状に分布しているハローと呼ばれる成分がある.ハローの星もランダムな運動をしている.バルジとハローをあわせて回転楕円体成分と呼ぶことがあり,どちらも比較的古い星が主体となっている.たとえば球状星団は年齢の古い星団であるが,おもにハローに分布している.これに対して比較的若い星団であるが,おもに円盤に分布している.

渦巻銀河では,早期型(Sa)から晩期型(Sc)に向かうに従い,次に示すように性質が変化する.

\begin{enumerate}[(1)]
	\item 円盤の明るさに対するバルジの明るさの比が小さくなる.
	\item 渦巻腕の巻き込みの度合いが緩やかになる.
	\item 円盤で巨大な電離水素領域(HII領域)や若い明るい星と星団が目立ってくる.
	\item 星に対するガスや塵の相対質量が大きくなる.
\end{enumerate}

渦巻銀河は,大きく分けて普通の渦巻銀河と棒渦巻銀河に分かれる.棒渦巻銀河の割合は,およそ20\%--30\%であるが,詳しく調べると大部分の渦巻銀河には多少とも棒状構造が見られ,顕著な棒状構造のあるものを棒渦巻銀河と呼んでいる.したがって棒渦巻銀河と普通の渦巻銀河に大きな性質の差があるとは考えない方が良い.棒状構造は銀河同士の相互作用などでも生じると考えられるが,銀河に内在する要因によって次第に成長するという説もあり,成因はまだ十分に理解されていない.