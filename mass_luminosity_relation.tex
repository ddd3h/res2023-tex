\section{質量光度関係}

天体の質量と光度との比をを質量--光度比といい,通常太陽の値で規格化した値として$M/L$と表す.すなわち,銀河の質量を$M_g$,光度を$L_g$とすると
\begin{equation}
	M/L = \frac{M_g/M_\odot}{L_g/L_\odot}
\end{equation}
である.星の場合,光度,質量とも太陽と同じでなら$M/L \sim 1$であり,太陽より軽くて暗い星では1より大きく,太陽より重くて明るいほしでは1より小さいことが知られている.銀河の場合はたくさんの星やガスの集まりとして銀河円盤が形成されるので$M/L$はその総和として決められる.一般的には,銀河内のガスの総質量は,星の総和に比べて小さいことが多く,銀河円盤の$M/L$は円盤内の平均的な星の質量と光度によって決まっている.また,暗黒物質のように光を一切出さない物質の場合は$M/L \sim \infty$となる。実際の銀河全体の$M/L$は,円盤とハローの総和から求まるので,銀河について$M/L$比を求めることで,銀河を形成している物質は何が支配的か(星か暗黒物質か)を知ることができる.銀河全体について$M/L$比を出す場合は,観測された銀河回転速度から銀河の質量$M$を求め,光学観測から決めた$L$と合わせて得られる.その際,銀河の距離が必要になり,ハッブル定数と銀河の後退速度とから求めるか,タリー・フィッシャー関係(Tully-Fisher relation)などの経験則を用いて決定する.多くの銀河が$M/L$比が10から20程度の値になることが知られている.


すなわち,恒星の質量を$M$,光度を$L$とすると
\begin{equation}
    \frac{L}{L_\odot} = \left(\frac{M}{M_\odot}\right)^{a}
\end{equation}
と表せる.指数$a$の値は恒星質量の範囲によって異なる値をとり,それぞれの範囲に対して以下の式でよく近似できる.

\begin{align}
	\frac{L}{L_\odot} \approx \begin{dcases}
		0.23 \left(\frac{M}{M_\odot}\right)^{2.3} & (M<0.43 M_\odot) \\
		\left( \frac{M}{M_\odot} \right)^4 & (0.43M_\odot < M <2 M_\odot) \\
		1.4 \left(\frac{M}{M_\odot}\right)^{3.5} & (2M_\odot < M < 20 M_\odot) \\
		32000 \frac{M}{M_\odot} & (55 M_\odot < M)
	\end{dcases}
\end{align}

ここで今回解析する天体は銀河であり,$10^{7}\--10^{13} M_\odot$であるからより$L \propto M$としてよい.


恒星の光度 (単位時間あたりに放射されるエネルギー) を決める重要な要素は、恒星全体でのエネルギー散逸率である。対流が存在しない場合、散逸は主に光子の拡散によって起きる。対流が無視できる領域である放射層におけるある半径 r での表面積にわたってフィックの法則を積分することで、外向きに流れる総エネルギー流を計算することが出来る。エネルギー保存の法則より、このエネルギーのフラックスは恒星の光度と等しくなる。 