\section{ビリアル半径 $R_\text{vir}$}

赤方偏移$z$においてビリアル平衡に達したダークマターハローの平均密度は:
\begin{align}
	\rho_\text{vir}(z) &= \rho_\text{cr} \Delta_\text{vir} \\
	&\simeq \num{1.8e-27} \left( \frac{\Delta_\text{vir}}{200}\right) \left(\frac{h}{0.7}\right)^2 E^2(z) \ \si{g. cm^{-3}}
\end{align}

ここでビリアル半径内の全質量(ビリアル質量)を$M_\text{vir}$を用いると
\begin{align}
	R_\text{vir} &= \left( \frac{3 M_\text{vir}}{4 \pi \rho_\text{vir}(z)} \right)^{1/3} \\
	&\simeq 2.1 \left( \frac{M_\text{vir}}{10^{15} M_\odot} \right)^{1/3} \left( \frac{\Delta_\text{vir}}{200} \right)^{-1/3} \left( \frac{h}{0.7} \right)^{-2/3} E^{-2/3}(z) \quad \si{Mpc}
\end{align}
となり、観測される銀河団のサイズと質量等の関係を近似的に再現する。

ここでは近傍宇宙を考えているので、$z=0$では$E(z) = 1$であり、$\Delta_\text{vir} = 200$のときのビリアル半径を$R_{200}$とすると
\begin{align}
	R_{200} \simeq 2.1 \left( \frac{M_\text{vir}}{10^{15} M_\odot} \right)^{1/3} \left( \frac{h}{0.7} \right)^{-2/3} \quad \si{Mpc}
\end{align}

\begin{algorithmic}
	\Function{Solve\_Virial\_Mass}{$\text{radius}, \text{mass}, \text{density\_DM}, \text{density\_total}$}
	\State $\text{valid\_indices} \gets \text{indices of } \text{radius} \text{ where } \text{total\_mass} \text{ is not } \infty$
	\State $\text{radius} \gets \text{radius}[ \text{valid\_indices}]$
	\State $\text{total\_mass} \gets \text{total\_mass}[ \text{valid\_indices}]$
	
	\State $\text{sorted\_radius}, \ \text{sorted\_total\_mass} \gets \text{sort } \text{radius}, \ \text{total\_mass} \text{ based on } \text{radius}$
	
	\State $\text{cum\_mass} \gets \text{cumulative sum of } \text{sorted\_total\_mass}$
	
	\State $h \gets 0.6774$
	
	\State $\text{virial\_radius} \gets 2.1 \times \left(\frac{\text{cum\_mass} \times 10^{10}}{10^{15}}\right)^{\frac{1}{3}} \times \left(\frac{h}{0.7}\right)^{-\frac{2}{3}}$
	
	\State $\text{min\_index} \gets \text{index of the element in } \text{radius}$ \\
	\qquad \qquad \qquad \qquad \qquad$\text{ with the smallest value of } (\text{radius} - \text{virial\_radius})^2$
	\State $ r_{200} \gets \text{radius}[\text{min\_index}]$
	
	\State \textbf{return} $r_{200}$
	\EndFunction
\end{algorithmic}