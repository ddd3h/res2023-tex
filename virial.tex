\section{ビリアル半径 $R_\text{vir}$}

物質の分布に対して球対称性を仮定する.この場合,球の中心となる地点から半径$R$内に存在する物質の質量を$M$とすると,運動方程式
\begin{equation}
	\dv[2]{R}{t} = - \frac{GM}{R^2}
\end{equation}
が成り立つ.ここで$R$は宇宙膨張と切り離された静止座標系での長さで,ハッブル半径よりも十分に小さい(つまり,ニュートン力学が適用できる)とする.この方程式の束縛解(有界な解)は,媒介変数$\theta$を用いて
\begin{equation}
	R=\frac{GM}{C}(1-\cos\theta),\ t = \frac{GM}{C^{3/2}}(\theta - \sin\theta) \label{eq:2158}
\end{equation}
と表される.ここで$C$は初期条件により決まる定数である.これらより,半径$R$内の物質密度は次式で表せされる.
\begin{equation}
	\rho(<R,t) = \frac{3M}{4\pi R^3} = \frac{1}{6\pi Gt^2} \left[ \frac{9(\theta - \sin\theta)^2}{2(1-\cos\theta)^3} \right] \label{eq:2159}
\end{equation}
さらに,物質優勢期における宇宙の臨界質量密度$ \rho_\text{cr}(t) \simeq \overline{\rho}_\text{m}(t) \simeq 1/{6\pi Gt^2}$が与えられることを用いると,この時期における球対称領域の「密度超過」は,初期条件によらず
\begin{equation}
	\Delta(t) \equiv \frac{\rho(<R,t)}{\rho_\text{cr}(t)} = \left[\frac{9(\theta - \sin\theta)^2}{2(1-\cos\theta)^3}\right]
\end{equation}
と表される.以下では,この時間発展の様子をみる.

まず,式\eqref{eq:2158}より$t$は$\theta$の単調増加関数であり,$\theta \ll 1$は宇宙初期に対応する.このとき式\eqref{eq:2158}のマクローリン展開は
\begin{equation}
	R \simeq \frac{GM}{C} \left( \frac{\theta^2}{2!} - \frac{\theta^4}{4!} + \cdots \right), \ t \simeq \frac{GM}{C^{3/2}} \left( \frac{\theta^3}{3!} - \frac{\theta^5}{5!} + \cdots \right)
\end{equation}
であり,この時期には$R$も$\theta,t$とともに単調増加する.また,式\eqref{eq:2159}は
\begin{equation}
	\rho(<R, t) \simeq \frac{1}{6 \pi G t^2} \left[ 1 + \frac{3C}{20}\left( \frac{6t}{GM} \right)^{2/3} + \cdots \right] \label{eq:2161}
\end{equation}
と近似され,式\eqref{eq:2161}のそれぞれ第一項から
\begin{equation}
	\frac{\dot{R}}{R} = \frac{2}{3t}
\end{equation}
が物質優勢期のハッブルパラメータ$H(t) \simeq 2/3t$と同じ形をしており,球形領域が当初は宇宙とともに膨張していたことを示す.また$\theta = \pi$に達すると,半径$R$は増加から減少に転じ,球形領域は収縮を始める.この時点の物理量を添字ta (転回を意味する英語"turn-around"の略)で表す.このときのポテンシャルと運動エネルギーの和は$U_\text{ta}+K_\text{ta} = U_\text{ta}$がいえる.

ここでビリアル定理を銀河内の物質に対して適用する.

一様球に対するエネルギー保存則
\begin{equation}
	K + \overline{U} = U_\text{ta}
\end{equation}
および自己重力ポテンシャル$U \propto 1/R$と合わせると,平衡下での平均半径に対して
\begin{equation}
	\overline{R} = \frac{1}{2}R_\text{ta}
\end{equation}
が得られる.このような状態は慣用的に「ビリアル平衡(virial equilibrium)」と呼ばれ,これに達した物理量は
\begin{align}
	\overline{t} \simeq t(\theta = 2\pi) = 2t_\text{ta},\\
	\overline{\Delta} = 18 \pi^2 \simeq 178,
\end{align}
となる.

赤方偏移$z$においてビリアル平衡に達したダークマターハローの平均密度は:
\begin{align}
	\rho_\text{vir}(z) &= \rho_\text{cr} \Delta_\text{vir} \\
	&\simeq \num{1.8e-27} \left( \frac{\Delta_\text{vir}}{200}\right) \left(\frac{h}{0.7}\right)^2 E^2(z) \ \si{g. cm^{-3}}
\end{align}
と表せる。

ここでビリアル半径内の全質量(ビリアル質量)を$M_\text{vir}$を用いると
\begin{align}
	R_\text{vir} &= \left( \frac{3 M_\text{vir}}{4 \pi \rho_\text{vir}(z)} \right)^{1/3} \\
	&\simeq 2.1 \left( \frac{M_\text{vir}}{10^{15} M_\odot} \right)^{1/3} \left( \frac{\Delta_\text{vir}}{200} \right)^{-1/3} \left( \frac{h}{0.7} \right)^{-2/3} E^{-2/3}(z) \quad \si{Mpc}
\end{align}
となり、観測される銀河団のサイズと質量等の関係を近似的に再現する。

ここでは近傍宇宙を考えているので、$z=0$では$E(z) = 1$であり、$\Delta_\text{vir} = 200$のときのビリアル半径を$R_{200}$とすると次の式が成り立つ:
\begin{align}
	R_{200} \simeq 2.1 \left( \frac{M_\text{vir}}{10^{15} M_\odot} \right)^{1/3} \left( \frac{h}{0.7} \right)^{-2/3} \quad \si{Mpc}
\end{align}


%\begin{algorithmic}
%	\Function{Solve\_Virial\_Mass}{$\text{radius}, \text{mass}, \text{density\_DM}, \text{density\_total}$}
%	\State $\text{valid\_indices} \gets \text{indices of } \text{radius} \text{ where } \text{total\_mass} \text{ is not } \infty$
%	\State $\text{radius} \gets \text{radius}[ \text{valid\_indices}]$
%	\State $\text{total\_mass} \gets \text{total\_mass}[ \text{valid\_indices}]$
%	
%	\State $\text{sorted\_radius}, \ \text{sorted\_total\_mass} \gets \text{sort } \text{radius}, \ \text{total\_mass} \text{ based on } \text{radius}$
%	
%	\State $\text{cum\_mass} \gets \text{cumulative sum of } \text{sorted\_total\_mass}$
%	
%	\State $h \gets 0.6774$
%	
%	\State $\text{virial\_radius} \gets 2.1 \times \left(\frac{\text{cum\_mass} \times 10^{10}}{10^{15}}\right)^{\frac{1}{3}} \times \left(\frac{h}{0.7}\right)^{-\frac{2}{3}}$
%	
%	\State $\text{min\_index} \gets \text{index of the element in } \text{radius}$ \\
%	\qquad \qquad \qquad \qquad \qquad$\text{ with the smallest value of } (\text{radius} - \text{virial\_radius})^2$
%	\State $ r_{200} \gets \text{radius}[\text{min\_index}]$
%	
%	\State \textbf{return} $r_{200}$
%	\EndFunction
%\end{algorithmic}