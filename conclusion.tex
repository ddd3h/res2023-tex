\documentclass[main.tex]{subfiles}
\begin{document}
	\chapter{まとめと展望}
	今回主に解析をしたsubhalo342447は動径方向の元素分布や温度分布から重力崩壊型の超新星爆発に由来し,形成過程において他のsubhaloと衝突した可能性があることが示された.またアウトフローと太陽組成比に何らかの因果関係がある可能性があることも分かった.
	
	今後はsubhalo342447を実際にXRISM衛星\footnote{
	XRISM(クリズム)は、星や銀河の間を吹き渡る高温プラズマを観測するX線天文衛星です。2016年に姿勢制御系の不具合のため短期間で運用終了したX線天文衛星「ひとみ」の後継機として開発され,JAXA宇宙科学研究所,NASA,ESAといった国内外の研究機関が共同開発した衛星である.2023年9月7日に打ち上げられ,約4ヶ月にわたる立ち上げ運用を終えて,本格観測を開始した.
	
	XRISMは,搭載する最新のX線分光装置やX線撮像装置で宇宙を観測し,高温プラズマの速度や化学組成を調べる.X線の精密分光撮像によって,星や銀河の中でつくりだされる物質やエネルギーの流転を調べ,天体の進化を解明する.
	
	XRISMは,これまでにない高い精度でX線を検出することができ,データから天体の温度や速度などの状態を分析することで,ブラックホールや暗黒物質=ダークマターなど宇宙の謎の解明につながる可能性があると期待されている.}で観測した場合,どのようなデータが得られるのかを調査したい.またsubhalo342447のような銀河を実際にアウトフローや物質構造を推定するためにはどれぐらいの分解能を有する検出器が必要で,なおかつ必要な観測時間はどれほどであるかを計算することで今後の検出器の開発目標値としての指針を与えることができるであろう.

	さらに,subhalo342447以外のsubhaloについても同様の解析を行い,複数のデータを比較することで一般的な傾向や特異性を把握していきたい。

	このような詳細な解析と観測により,銀河の形成過程や物質の分布,そしてアウトフローと太陽組成比の関係性についての理解が深まり,宇宙の起源や進化についての新たな知見を提供し,我々の宇宙観を一段と進化させる可能性があるだろう.
	
\end{document}