\section{太陽組成比 (solar abundance)}

我々の宇宙や太陽がどういう組成でできており,元素ごとの存在比で示した量として太陽組成(宇宙組成)があり,英語でsolar abundanceやabundance (Abundance of chemical elementsの略) と言われることが多い.この太陽組成(宇宙組成)は太陽光球の分光観測で得られた元素組成を指すだけの時もあれば,隕石(コンドライト)の分析値を合わせて,隕石が太陽系全体の元素存在度(宇宙組成比)をよく近似していると仮定して,太陽組成ということもある.

太陽は始原的(太陽系生成前の環境)な元素分布を保持していると考えられており,隕石の中でも始原的な隕石は太陽系形成時のタイムカプセルのようにそのときの元素分布を保持していると考えると両者には近い関係が期待できる.太陽系の素は,宇宙が137億年前にビックバンで生成して,太陽系ができた45億年前に,多くの星が一生を終えて、様々な元素が散りばめられ混ざった状態から太陽が生まれたと考えられているため、宇宙の平均的な組成は太陽組成に近いと考えられます