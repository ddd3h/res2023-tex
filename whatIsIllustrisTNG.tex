\section{IllustrisTNGのデータの取り扱い}
Illustris-TNGは、銀河形成をシミュレーションする大規模な磁気流体力学シミュレーションのシリーズで、2017年に発表されたIllustrisプロジェクトの後継として、より高解像度とより正確な物理モデルを備えている。

Illustris-TNGは、暗黒物質、ガス、星の3つの物質成分をシミュレートします。暗黒物質は、銀河の重力を支配する仮説上の粒子で、ガスは銀河の星形成の材料となり、星は銀河の光源となります。

Illustris-TNGは、銀河の形成と進化のさまざまな側面をシミュレートします。具体的には、以下のようなことをシミュレートします。

* 銀河の形成と進化
* 銀河の構造と組成
* 銀河の相互作用
* 銀河の進化の歴史

Illustris-TNGは、銀河形成の理解を深めるために重要なツールです。シミュレーション結果は、銀河の観測結果と比較することで、銀河形成の物理モデルを検証し、改善するために使用できます。

Illustris-TNGは、3つの異なる解像度で実行されています。

* TNG50:50億パーセク(約160億光年)の体積を、約5000万個の粒子でシミュレートします。
* TNG100:100億パーセク(約320億光年)の体積を、約10億個の粒子でシミュレートします。
* TNG300:300億パーセク(約960億光年)の体積を、約100億個の粒子でシミュレートします。

Illustris-TNGのデータは、一般に公開されています。このデータは、銀河形成の研究を行う研究者や学生が使用できます。