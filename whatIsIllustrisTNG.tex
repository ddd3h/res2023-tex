\section{IllustrisTNGのデータの取り扱い}
%Illustris-TNGは、銀河形成をシミュレーションする大規模な磁気流体力学シミュレーションのシリーズで、2017年に発表されたIllustrisプロジェクトの後継として、より高解像度とより正確な物理モデルを備えている。
%
%Illustris-TNGは、暗黒物質、ガス、星の3つの物質成分をシミュレートします。暗黒物質は、銀河の重力を支配する仮説上の粒子で、ガスは銀河の星形成の材料となり、星は銀河の光源となります。
%
%Illustris-TNGは、銀河の形成と進化のさまざまな側面をシミュレートします。具体的には、以下のようなことをシミュレートします。
%
%* 銀河の形成と進化
%* 銀河の構造と組成
%* 銀河の相互作用
%* 銀河の進化の歴史
%
%Illustris-TNGは、銀河形成の理解を深めるために重要なツールです。シミュレーション結果は、銀河の観測結果と比較することで、銀河形成の物理モデルを検証し、改善するために使用できます。
%
%Illustris-TNGは、3つの異なる解像度で実行されています。
%
%* TNG50:50億パーセク(約160億光年)の体積を、約5000万個の粒子でシミュレートします。
%* TNG100:100億パーセク(約320億光年)の体積を、約10億個の粒子でシミュレートします。
%* TNG300:300億パーセク(約960億光年)の体積を、約100億個の粒子でシミュレートします。
%
%Illustris-TNGのデータは、一般に公開されています。このデータは、銀河形成の研究を行う研究者や学生が使用できます。

Illustris-TNGはVolker Springelが率いて作られた最先端の宇宙論的銀河形成シミュレーションで、銀河形成を促進する様々な物理過程を考慮しながら、ビッグバン直後から現在までの模擬宇宙の広い範囲をシミュレーションしている。シミュレーションデータはTNG50、TNG100、TNG300の3つが存在し、それぞれ空間体積が\SI{50}{Mpc}、\SI{100}{Mpc}、\SI{300}{Mpc}の立法体内でシミュレーションを行っている。最も大きいTNG300は、銀河団などの珍しい天体の解析が可能であり、最大の銀河サンプルが得られる。一方、体積の小さいTNG50では、希少天体のサンプリングは比較的限定されるが、TNG300に比べ質量分解能は数百倍高く、銀河の構造的性質、銀河周辺のガスの詳細な構造、物理モデルの収束性などをより詳細に調べることができる。そこで本研究ではTNG50-1を利用して解析を行う。

\begin{table}[htbp]
	\centering
	\caption{}
	\label{tab:}
	\begin{tabular}{ccccc}
		\toprule
		&  & TNG50 & TNG100 & TNG300 \\
		\midrule
		Volume & [\si{Mpc^3}] & $51.7^3$ & $110.7^3$ & $302.6^3$ \\
		$L_\text{box}$& [\si{Mpc}/h] & 35 & 75  & 205 \\
		$N_\text{GAS}$& - & $2160^3$ & $1820^3$ & $2500^3$ \\
		$N_\text{DM}$& - & $2160^3$ & $1820^3$ & $2500^3$ \\
		$N_\text{TR}$& - & $2160^3$ & $2\times1820^3$ & $2500^3$ \\
		$m_\text{baryon}$& [\si{M_\odot}] & \num{8.5e+4} & \num{1.4e+4} & \num{1.1e+7} \\
		$m_\text{DM}$& [\si{M_\odot}] & \num{4.5e+4} & \num{7.5e+4} & \num{5.9e+7} \\
		$\epsilon_\text{gas,min}$ & [pc] & 74 & 185 & 370 \\
		$\epsilon_\text{DM,*}$ & [pc] & 288 & 740 & 1480 \\
		\bottomrule
	\end{tabular}
\end{table}

\clearpage

\begin{verbatim}
	TNG50-1
	├── output
	│ ├── groups_099
	│   │   ├── fof_subhalo_tab_099.0.hdf5
	│   │   └── fof_subhalo_tab_099.99.hdf5
	│   └── snapdir_099
	│       ├── snap_099.0.hdf5
	│       └── snap_099.99.hdf5
	└── postprocessing
	└── offsets
	└── offsets_099.hdf5
\end{verbatim}