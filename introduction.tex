\chapter{はじめに}

宇宙背景放射を観測した最新の人工衛星Planckのデータを解析すると、宇宙のエネルギー密度は、通常物質(バリオン)が4.9\%、ダークマターが26.8\%、ダークエネルギーが68.3\%であることがわかっている。このことからも宇宙の大局的構造進化は、ダークエネルギーとダークマターが担っていると言われている。しかし、バリオンは元素の元となり、天体形成、超新星爆発やブラックホールなど、宇宙のさまざまな事象を引き起こす主役であり、宇宙の構造形成と進化を理解する上で我々が現在のところ直接観測できるのはバリオンだけである。

低赤方偏移で観測されたバリオンは、電離放射線、銀河形成、星形成といった大規模な構造によって推定が非常に難しく、不正確になる。そのため、このバリオンの大半が未発見である。この問題は「ミッシングバリオン問題」と呼ばれ、宇宙物理学上に残された重要な問題の一つである。

% この未発見のバリオン、「ダークバリオン」を探査するためには、X線帯域での観測が有効であり、各国で203--40年代のX線バリオン探査計画の検討(日本のSuper DIOS計画、米国のLine Emission Mapper計画、中国のHot Univers Suervey計画など)が進んでいる。

「ダークバリオン」の多くは銀河間空間に分布していると考えられ(Warm-Hot Intergalactic Medium: WHIMと呼ばれる)、これらは個々の銀河周辺(~10 kpc), 銀河の大集団である銀河団周辺(~1 Mpc),銀河団をもつなぐ宇宙の大規模構造(~100 Mpc)、と宇宙の各階層構造に広く分布していると考えれる。宇宙の構造形成を明らかにするためには、各階層でのバリオンの分布を定量的に調べる必要がある。私はこの中でも、我々の銀河系のような渦巻き銀河や楕円銀河周辺の物質構造について着目している。これらは特に、Circum Galactic Medium (CGM)と呼ばれ、可視光や電波でのスタッキング観測も報告されているが(ex., Tanimura et al. MNRAS, 2019)、ガス構造や元素分布の解明には至っていない。特に、銀河内で生成された元素がどのように銀河間空間に供給されたのか、そのメカニズムに私は着目している。最近、我々の銀河系内のX線観測から、eROSITAバブルと呼ばれる銀河中心方向から延びるX線で明るい構造において、アルファ元素の比率が太陽組成よりも高いという報告もなされている(Gupta et al. Nature Astro., 2023)。この結果は、一般に銀河風と呼ばれる大量の重力崩壊型超新星爆発により銀河内の(元素を含む)ガスが銀河間空間に放出される現象を示唆しているが、系統誤差も多く結論づけるには尚早である。一方で、銀河中心にある超巨大ブラックホール(活動銀河中心核: AGN)によるジェットに付随するような構造であれば、ガスの速度構造や元素の組成が太陽組成に近いことなどが予想される。

上記議論に決着をつけるためには、CGMのX線直接観測が必須であるが、次世代衛星での観測を待つ必要がある。しかし、近年進歩が著しい宇宙論的シミュレーションデータベースを用いて予測することは可能である。宇宙論的シミュレーションは全世界に公開されているものも多く、Illustris-TNG(Phillepich et al. 2018)は次世代衛星の検出器感度の評価にも多く使われている。私は、Illustris-TNG のスナップショットデータの近傍渦巻き銀河のカタログを用いて、渦巻き銀河の周りのガスの速度と元素組成を系統的に調べることを考えている。データ容量も大きいが、独自にメモリ展開手法などを開発して解決しており、これまで行ってきた私のコード開発の経験を活かしデータ解析を進めている。また、光学的に薄いプラズマモデルを過程して、シミュレーションデータをもとにした模擬X線スペクトルを作成し、実際にどのように将来衛星で観測するべきかの観測戦略の検討もすることを考えている。2023年には高いエネルギー分光能力を実現するX線分光撮像衛星「XRISM」の打ち上げが予定されており、我々の銀河系に付随するガスの高精度の分光観測データとの比較も行いたいと考えている。

\section{IllustrisTNG}

\section{ビリアル半径}

\section{outflowに関して}