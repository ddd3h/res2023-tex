\chapter{はじめに}

宇宙背景放射を観測した最新の人工衛星Planckのデータを解析すると、宇宙のエネルギー密度は、通常物質(バリオン)が4.93\%、ダークマターが26.43\%、ダークエネルギーが68.64\%であることがわかっている\citep{aghanim_planck_2020}。このことからも宇宙の大局的構造進化は、ダークエネルギーとダークマターが担っていると言われている。しかし、バリオンは元素の元となり、天体形成、超新星爆発やブラックホールなど、宇宙のさまざまな事象を引き起こす主役であり、宇宙の構造形成と進化を理解する上で我々が現在のところ直接観測できるのはバリオンだけである。

特に近傍宇宙におけるバリオンの構造進化は観測的にも不明確なことが多い.そのため現在の宇宙では,バリオンの大半が未発見である\citep{shull_baryon_2012}.この問題は「ミッシングバリオン問題」と呼ばれ,宇宙物理学に残された重要な課題である.

銀河サーベイによってバリオンの約10\%が銀河、銀河群、銀河団などの天体に存在することがわかり、特に過去15年間で、銀河間物質(Inter Galactic Medium, IGM)、銀河ハロー、銀河周辺物質(Circum Galactic Medium, CGM)\footnote{銀河周辺物質(Circum Galactic Medium, CGM)には銀河から吹き出された物質のことを指す場合や、ビリアル半径内の物質のことを指す場合など文脈によって複数の意味を持つ。}にかなりの量のガスが存在することがわかった。残りの80\%--90\%のうち約半数は、IGMや中高温銀河間物質(Warm-Hot Intergalactic Medium, WHIM)に存在すると言われている\citep{shull_baryon_2012,danforth_low-z_2008}。

WHIMはほとんど完全電離した $10^5\--10^7$ K で非常に希薄なガスであり、遠紫外線や軟X線を出すが、観測で捉えるのが非常に難しい。それゆえ、未発見のバリオンの大部分が存在すると考えられている。WHIMは、個々の銀河周辺($\sim\SI{10}{kpc}$),銀河の大集団である銀河団周辺($\sim\SI{1}{Mpc}$),銀河団をもつなぐ宇宙の大規模構造($\sim\SI{100}{Mpc}$),と宇宙の各階層構造に広く分布していると考えれる.各階層でのバリオンの分布を定量的に調べ,宇宙論的進化を明らかにすることで構造形成を支配するダークマターに新たな制限を与えることができる.

本研究では,宇宙の階層構造の中でも,我々の銀河系のような渦巻き銀河や楕円銀河周辺の物質構造について着目した.可視光や電波でのスタッキング観測も報告されているが(ex., Tanimura et al. MNRAS, 2019),ガス構造や元素分布の解明には至っていない.特に,銀河内で生成された元素がどのように銀河間空間に供給されたのか,そのメカニズムに着目してその解明を目指す.